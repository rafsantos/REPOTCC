%%%%%%%%%%%%%%%%%%%%%%%%%%%%%%%%%%%%%%%%%%%%%%%%
% Classe de documento
%%%%%%%%%%%%%%%%%%%%%%%%%%%%%%%%%%%%%%%%%%%%%%%%
\documentclass[12pt]{abntex2}
\usepackage[utf8]{inputenc}
\usepackage{natbib}
\usepackage{graphicx}

\autor{Rafael Lourenço dos Santos}
\titulo{TCC do Rafael}
\data{2023}
\local{Rio de Janeiro}

\preambulo{Trabalho de conclusão de curso do Rafael}
\orientador{Michel Cambrainha}
\tipotrabalho{monografia}

%%%%%%%%%%%%%%%%%%%%%%%%%%%%%%%%%%%%%%%%%%%%%%%%
% Início do corpo do texto
%%%%%%%%%%%%%%%%%%%%%%%%%%%%%%%%%%%%%%%%%%%%%%%%
\begin{document}
\imprimircapa
\imprimirfolhaderosto
\section{Justificativa}
Uma novidade que a aprovação da Base Nacional Comum Curricular (BNCC) trouxe são as habilidades denominadas de "Pensamento Computacional", a cargo do núcleo de Matemática e suas Tecnologias. Com isso, torna-se fundamental discutir os recursos didáticos que os docentes utilizarão para atingir este objetivo. A própria escolha pelo nome pensamento computacional indica que a habilidade que se deseja do estudante não é a de uma capacitação em operar computadores, nem a de sintaxe de uma linguagem de programação específica. O esperado é que o aluno construa sólido raciocínio lógico, compreensão e modelagem de problemas e a elaboração, automatização, interpretação e comparação de algoritmos aplicáveis em diversos contextos. Algo a se destacar, é que o uso de computadores no ensino neste período esteve restrito a estas aulas complementares, fora do currículo obrigatório. Também neste período, o uso de computadores no ensino era de forma independe das outras disciplinas, que não foram influenciadas pela massificação da tecnologia e sua chegada ao ambiente escolar.
Também, mesmo que o custo de aquisição de computadores viesse diminuindo, ainda era considerado elevado. Por isso, poucas instituições de ensino dispunham de recursos para disponibilizar um laboratório de informática onde mesmo estas atividades de ensino de habilidades operacionais pudesse ser realizada. 
\\
Historicamente, com a popularização dos computadores e com sua adoção em larga escala no ambiente de trabalho, as instituições de ensino tiveram a necessidade de preparar os estudantes para que quando egressos estivessem preparados para desempenhar suas atividades neste novo mundo do trabalho. Esta capacitação concentrou-se em desenvover habilidades operacionais capacitando o indivíduo em operar tarefas específicas utilizando um computador, em situações que ele poderia vivenciar quando fosse trabalhar. Por exemplo, elaborar documentos usando um processador de texto ou montar planilhas com aplicativos específicos. Em seguida, com a evolução da Intenet a partir dos anos 90, as escolas que já tinham 



\end{document}
