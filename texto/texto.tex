%%%%%%%%%%%%%%%%%%%%%%%%%%%%%%%%%%%%%%%%%%%%%%%%
% Classe de documento
%%%%%%%%%%%%%%%%%%%%%%%%%%%%%%%%%%%%%%%%%%%%%%%%
\documentclass[12pt]{abntex2}
\usepackage[utf8]{inputenc}
\usepackage{natbib}
\usepackage{graphicx}

\autor{Rafael Lourenço dos Santos}
\titulo{TCC do Rafael}
\data{2023}
\local{Rio de Janeiro}

\preambulo{Trabalho de conclusão de curso do Rafael}
\orientador{Michel Cambrainha}
\tipotrabalho{monografia}

%%%%%%%%%%%%%%%%%%%%%%%%%%%%%%%%%%%%%%%%%%%%%%%%
% Início do corpo do texto
%%%%%%%%%%%%%%%%%%%%%%%%%%%%%%%%%%%%%%%%%%%%%%%%
\begin{document}
\imprimircapa
\imprimirfolhaderosto
\section{Justificativa}

Uma inovação que a aprovação da Base Nacional Comum Curricular (BNCC) trouxe é a inserção de disciplinas de Pensamento Computacionalhypara o currículo de Matemática é  



\end{document}
